\documentclass[a4paper,10pt]{article}

\usepackage{appendixnumberbeamer}
\usepackage{booktabs}
\usepackage[scale=2]{ccicons}
\usepackage{pgfplots}
\usepackage{xspace}
\usepackage{bookmark}
\usepackage{amssymb}
\usepackage{mathtools}
\usepackage[normalem]{ulem}
\usepackage[T1]{fontenc}
\usepackage[sfdefault,book]{FiraSans}
\usepackage{FiraMono}
\usepackage{fontawesome}
\usepackage{listings}
\usepackage{hyperref}
\usepackage[10pt]{moresize}

\renewcommand*\familydefault{\ttdefault}

\renewcommand{\thefootnote}{\fnsymbol{footnote}}
\renewcommand{\MintedPygmentize}{/home/thiago/.local/bin/pygmentize}

\definecolor{light-gray}{gray}{0.95}
\renewcommand\lstlistingname{Código}
\DeclareCaptionFormat{listing} {
  \parbox{\textwidth}{\hspace{-0.2cm}#1#2#3}
}
\DeclareCaptionFont{black}{\color{black}}
\captionsetup[lstlisting]{
  format=listing,
  labelfont=black,
  textfont=black,
  singlelinecheck=true,
  margin=0pt,
  font={tt,footnotesize,bf}
}

\lstset{
  basicstyle=\footnotesize\ttfamily,
  escapeinside={\%*}{*)},
  mathescape=true,
  showspaces=false,
  showtabs=false,
  showstringspaces=false,%
  % backgroundcolor=\color{light-gray},
  rulesepcolor=\color{black},
  frame=shadowbox,
  literate=
  {á}{{\'a}}1 {é}{{\'e}}1 {í}{{\'i}}1 {ó}{{\'o}}1 {ú}{{\'u}}1
  {Á}{{\'A}}1 {É}{{\'E}}1 {Í}{{\'I}}1 {Ó}{{\'O}}1 {Ú}{{\'U}}1
  {à}{{\`a}}1 {è}{{\`e}}1 {ì}{{\`i}}1 {ò}{{\`o}}1 {ù}{{\`u}}1
  {À}{{\`A}}1 {È}{{\'E}}1 {Ì}{{\`I}}1 {Ò}{{\`O}}1 {Ù}{{\`U}}1
  {ä}{{\"a}}1 {ë}{{\"e}}1 {ï}{{\"i}}1 {ö}{{\"o}}1 {ü}{{\"u}}1
  {Ä}{{\"A}}1 {Ë}{{\"E}}1 {Ï}{{\"I}}1 {Ö}{{\"O}}1 {Ü}{{\"U}}1
  {â}{{\^a}}1 {ê}{{\^e}}1 {î}{{\^i}}1 {ô}{{\^o}}1 {û}{{\^u}}1
  {Â}{{\^A}}1 {Ê}{{\^E}}1 {Î}{{\^I}}1 {Ô}{{\^O}}1 {Û}{{\^U}}1
  {Ã}{{\~A}}1 {ã}{{\~a}}1 {Õ}{{\~O}}1 {õ}{{\~o}}1
  {œ}{{\oe}}1 {Œ}{{\OE}}1 {æ}{{\ae}}1 {Æ}{{\AE}}1 {ß}{{\ss}}1
  {ű}{{\H{u}}}1 {Ű}{{\H{U}}}1 {ő}{{\H{o}}}1 {Ő}{{\H{O}}}1
  {ç}{{\c c}}1 {Ç}{{\c C}}1 {ø}{{\o}}1 {å}{{\r a}}1 {Å}{{\r A}}1
  {€}{{\euro}}1 {£}{{\pounds}}1 {«}{{\guillemotleft}}1
  {»}{{\guillemotright}}1 {ñ}{{\~n}}1 {Ñ}{{\~N}}1 {¿}{{?`}}1
}

\setminted{
  fontsize=\footnotesize,
  style=xcode,
  frame=single,
  framesep=3\fboxsep,
  labelposition=topline,
}


\title{Reavaliação}
\posttitle{\end{center}}

\begin{document}

\maketitle

\emergencystretch 3em

\begin{itemize}[itemsep=0em]
  \item Não use celular/computador e não converse com ninguém, a prova é individual.
  \item Sinta-se à vontade para tirar dúvidas (\textbf{razoáveis}) ou pedir esclarecimentos sobre as questões.
  \item Use \textbf{letra legível}! não posso dar nota para algo que não consigo ler.
  \item Lembre-se de \textbf{assinar seu nome nas suas folhas}. Se usar \textbf{mais de uma} folha, \textbf{enumere cada página}.
  \item \textbf{Seja organizado:} especifique número e letra da questão que você está respondendo e deixe um espaço entre as respostas, para não ficar tudo amontoado. Você pode pegar mais folhas, se precisar.
\end{itemize}


NOME: \rule{.85\textwidth}{0.1mm}

\begin{multicols*}{2}
\setlength{\leftmargini}{0pt}
\begin{enumerate}
  \item (3,0 pt) Use o princípio da indução para provar as afirmações a seguir.

  \begin{enumerate}
    \item (0,8 pt) $ 6^{n} + 4 $ é divisível por $ 5 $, para todo $ n \geq 0 $.
    \item (1,0 pt) $ 5^{2n+1} + 2^{2n+1} $ é divisível por $ 7 $, para todo $ n \geq 0 $.
    \item (1,2 pt)
    \[
      \sum_{i=1}^{n} \frac{1}{i(i + 1)} = \frac{n}{n + 1},
    \]

    para todo $ n \geq 1 $.
  \end{enumerate}

  \item (2,0 pt) Considere o pseudocódigo abaixo:

  \begin{minted}{c}
Para i de 0 até n - 1:
  Para j de i + 1 até n - 1:
    Comando
  \end{minted}

    \begin{enumerate}
      \item (0,5 pt) Expresse como um somatório a quantidade de vezes que a linha \texttt{Comando} é executada.
      % \sum_{i = 0}^{n - 1} \sum_{j = i + 1}^{n - 1} 1
      \item (1,0 pt) Simplifique o somatório até chegar a uma fórmula em função de n. Para isso, use a expressão abaixo:
      \[
        \sum_{i = 0}^{n - 1} i = \frac{n(n - 1)}{2}
      \]
      % \frac{n(n - 1)}{2}

      \item (0,5 pt) De acordo a fórmula, qual é o tempo de execução de pior caso em notação Big-Oh?
      % O(n^2)
    \end{enumerate}

  \item (1,5 pt) Ordene as funções a seguir por sua dominância, da mais dominante para a menos dominante:

  \begin{itemize}[noitemsep]
    \item $ n^3 $
    \item $ n $
    \item $ \sqrt{n} $
    \item $ n \log{n} $
    \item $ \log{n} $
    \item $ n^2 $
    \item $ n! $
    \item $ c^n $
    \item $ 1 $
  \end{itemize}
  % n! >> c^n >> n^3 >> n^2 >> n \log{n} >> n >> \sqrt{n} >> \log{n} >> 1

  \item (1,5 pt) Mostre, usando as definições de $ O(f(n)) $ e $ \Omega(f(n)) $ que:

  \begin{enumerate}
    \item Se $ f(n) = O(g(n)) $ e $ g(n) = O(h(n)) $, então $ f(n) = O(h(n)) $
    % f(n)     <= c_1 g(n)
    % g(n)     <= c_2 h(n)
    % c_1 g(n) <= c_1 c_2 h(n)
    % f(n)     <= c_1 c_2 h(n)
    % f(n)     <= c_3 h(n), c_3 = c_1 c_2
    \item Se $ f(n) = \Omega(g(n)) $, então $ g(n) = O(f(n)) $
    % f(n) >= c g(n)
    % (1/c) f(n) >= g(n)
    % g(n) <= (1/c) f(n)
    % g(n) <= c' f(n), c' = 1/c
  \end{enumerate}

  \vfill\null
  \columnbreak

  \item (1,0 pt) Relacione cada complexidade de tempo com uma operação em um algoritmo.

  \begin{enumerate}
    \item $ O(n^2) $
    \item $ O(1) $
    \item $ O(n!) $
    \item $ O(\log{n}) $
    \item $ O(n) $
  \end{enumerate}

  \begin{enumerate}
    \item [(\textcolor{white}{c})] Percorrer um array do início ao fim
    \item [(\textcolor{white}{a})] Extrair de um array um elemento de índice x
    \item [(\textcolor{white}{d})] Percorrer uma matriz do início ao fim
    \item [(\textcolor{white}{e})] Gerar todas as permutações de um conjunto de dados
    \item [(\textcolor{white}{b})] Fazer uma busca binária em um array ordenado
  \end{enumerate}
  % (e)
  % (b)
  % (a)
  % (c)
  % (d)

  \item (1,0 pt) Para cada par de funções, especifique se $ f(n) = O(g(n)) $ ou $ f(n) = \Omega(g(n)) $. Lembre-se das relações de dominância.

  \bigskip

  \setlength{\tabcolsep}{2pt}

  \begin{tabular}{@{\hskip 2mm}lll}
    (a) & $ f(n) = \sqrt{n}, $ & $ g(n) = \log{(n^2)} $ \\[0.4em]
    (b) & $ f(n) = 6n^2, $ & $ g(n) = n^2 \log{n} $ \\[0.4em]
    (c) & $ f(n) = n \log{n} + n, $ & $ g(n) = \log{n} $ \\[0.4em]
    (d) & $ f(n) = 10 n^2, $ & $ g(n) = 2^n $ \\[0.4em]
    (e) & $ f(n) = (\log{n})^2, $ & $ g(n) = \log{n} $
  \end{tabular}
  % Omega
  % O
  % Omega
  % O
  % Omega
\end{enumerate}
\end{multicols*}
\end{document}
