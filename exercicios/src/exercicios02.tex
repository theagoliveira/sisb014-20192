\documentclass[a4paper,10pt]{article}

\input{packages.tex}
\usetheme{metropolis}

\usepgfplotslibrary{dateplot}

\lstset{
  basicstyle=\ttfamily,
  mathescape
}

\title{Exercícios 02 (Grafos)}
\posttitle{\end{center}}

\begin{document}

\maketitle

\emergencystretch 3em

{\raggedright \textbf{Para todas as questões, assuma que as listas de adjacência estão ordenadas por ordem crescente alfabética/numérica}.}

\bigskip

\begin{multicols*}{2}
\setlength{\leftmargini}{0pt}
\begin{enumerate}
  % SKIENA, S. S. The Algorithm Design Manual, 2nd Edition -- 5.31
  \item Quais são as estruturas de dados utilizadas na BFS e na DFS? % Pilha e fila, respectivamente

  %%%%%%%%%%%%%%%%%%%%%%%%%%%%%%%%%%%%%%%%%%%%%%%%
  % BFS                                          %
  %%%%%%%%%%%%%%%%%%%%%%%%%%%%%%%%%%%%%%%%%%%%%%%%
  % CORMEN, T. H.; LEISERSON, C. E.; RIVEST, R. L.; STEIN, C. Introduction to Algorithms, 3rd Edition -- 22.2-1
  \item Obtenha os valores de distância e parentesco para a execução da BFS no grafo dirigido a seguir, usando o vértice 3 como inicial.
  % (1/0) (2/3)    (3/0)
  %       /        /  |
  %      /        /   |
  %     /        /    |
  % (4/2)-----(5/1) (6/1)
  %
  %  1  2  3  4  5  6
  % -1  4 -1  5  3  3

  \begin{center}
    \includegraphics[width=.8\linewidth]{graph2.png}
  \end{center}

  % CORMEN, T. H.; LEISERSON, C. E.; RIVEST, R. L.; STEIN, C. Introduction to Algorithms, 3rd Edition -- 22.2-2
  \item Obtenha os valores de distância e parentesco para a execução da BFS no grafo não-dirigido a seguir, usando o vértice u como inicial.
  %
  % (r/4)--(s/3)  (t/1)--(u/0)
  %   |      |    /      / |
  %   |      |   /      /  |
  %   |      |  /      /   |
  % (v/5)  (w/2)  (x/1)  (y/1)
  %
  %  r  s  t  u  v  w  x  y
  %  s  w  u -1  r  t  u  u

  \begin{center}
    \includegraphics[width=\linewidth]{graph3.png}
  \end{center}

  % CORMEN, T. H.; LEISERSON, C. E.; RIVEST, R. L.; STEIN, C. Introduction to Algorithms, 3rd Edition -- 22.2-5
  \item Usando a figura da questão anterior, mostre como diferenças na ordem interna de cada lista de adjacência podem alterar a árvore de parentesco gerada pela BFS.

  %%%%%%%%%%%%%%%%%%%%%%%%%%%%%%%%%%%%%%%%%%%%%%%%
  % DFS                                          %
  %%%%%%%%%%%%%%%%%%%%%%%%%%%%%%%%%%%%%%%%%%%%%%%%
  % CORMEN, T. H.; LEISERSON, C. E.; RIVEST, R. L.; STEIN, C. Introduction to Algorithms, 3rd Edition -- Figure 22.5
  \item Obtenha os valores de tempo e parentesco e as classificações das arestas para a execução da DFS no grafos dirigidos a seguir. Use o vértice s como inicial para o primeiro grafo e assuma que os vértices são processados em ordem alfabética no segundo grafo.
  % PRIMEIRO GRAFO
  %
  % Arestas
  % s - w Árvore
  % s - z Direta
  % t - u Árvore
  % t - v Direta
  % u - t Retorno
  % u - v Árvore
  % v - s Cruzada
  % v - w Cruzada
  % w - x Árvore
  % x - z Árvore
  % y - x Retorno
  % z - w Retorno
  % z - y Árvore
  %
  %    |   s   t   u   v   w   x   y   z
  % ---|--------------------------------
  % P  |  -1  -1   t   u   s   w   z   x
  % Ti |   1  11  12  13   2   3   5   4
  % To |  10  16  15  14   9   8   6   7
  %
  % (s (w (x (z (y y) z) x) w) s) (t (u (v v) u) t)
  %
  % SEGUNDO GRAFO
  %
  % Arestas
  % q - s Árvore
  % q - t Árvore
  % q - w Direta
  % r - u Árvore
  % r - y Cruzada
  % s - v Árvore
  % t - y Árvore
  % t - x Árvore
  % u - y Cruzada
  % v - w Árvore
  % w - s Retorno
  % x - z Árvore
  % y - q Retorno
  % z - x Retorno
  %
  %    |   q   r   s   t   u   v   w   x   y   z
  % ---|----------------------------------------
  % P  |  -1  -1   q   q   r   s   v   t   t   x
  % Ti |   1  17   2   8  18   3   4   9  13  10
  % To |  16  20   7  15  19   6   5  12  14  11
  %
  % (q (s (v (w  w) v) s) (t (x (z z) x) (y y) t) q) (r (u u) r)

  \begin{center}
    \includegraphics[width=\linewidth]{graph4.png}
  \end{center}

  \begin{center}
    \includegraphics[width=\linewidth]{graph5.png}
  \end{center}

  % CORMEN, T. H.; LEISERSON, C. E.; RIVEST, R. L.; STEIN, C. Introduction to Algorithms, 3rd Edition -- 22.3-3
  \item \xout{Mostre a estrutura de parênteses gerada pela DFS para o grafo da questão 2, começando a partir do vértice 1.}
  % ARESTAS
  % 1 - 2 Árvore
  % 1 - 4 Direta
  % 2 - 5 Árvore
  % 3 - 5 Cruzada
  % 3 - 6 Árvore
  % 4 - 2 Retorno
  % 5 - 4 Árvore
  % 6 - 6 Cruzada
  %
  %    |   1   2   3   4   5   6
  % ---|------------------------
  % P  |  -1   1  -1   5   2   3
  % Ti |   1   2   9   4   3  10
  % To |   8   7  12   5   6  11
  %
  % (1 (2 (5 (4 4) 5) 2) 1) (3 (6 6) 3)

  %%%%%%%%%%%%%%%%%%%%%%%%%%%%%%%%%%%%%%%%%%%%%%%%
  % ORDENAÇÃO TOPOLÓGICA                         %
  %%%%%%%%%%%%%%%%%%%%%%%%%%%%%%%%%%%%%%%%%%%%%%%%
  \item Faça uma ordenação topológica para cada grafo a seguir
  \bigskip

  % SKIENA, S. S. The Algorithm Design Manual, 2nd Edition -- 5.2
  \begin{center}
    \includegraphics[width=\linewidth]{graph1.png}
  \end{center}
  % H A B D E G I J C F

  % CORMEN, T. H.; LEISERSON, C. E.; RIVEST, R. L.; STEIN, C. Introduction to Algorithms, 3rd Edition -- 22.4-1
  \begin{center}
    \includegraphics[width=\linewidth]{graph6.png}
  \end{center}
  % p n o s m r y v x w z u q t

  %%%%%%%%%%%%%%%%%%%%%%%%%%%%%%%%%%%%%%%%%%%%%%%%
  % COMPONENTES FORTEMENTE CONECTADOS            %
  %%%%%%%%%%%%%%%%%%%%%%%%%%%%%%%%%%%%%%%%%%%%%%%%
  % CORMEN, T. H.; LEISERSON, C. E.; RIVEST, R. L.; STEIN, C. Introduction to Algorithms, 3rd Edition -- 22.5-2
  \item Identifique os componentes fortemente conectados no segundo grafo da questão 5 e mostre o grafo acíclico resultante.
  %    qty----r
  %   / | \  |
  % svw |  \ |
  %     xz   u
\end{enumerate}
\end{multicols*}
\end{document}
