\documentclass[a4paper,11pt]{article}

\input{packages.tex}
\usetheme{metropolis}

\usepgfplotslibrary{dateplot}

\lstset{
  basicstyle=\ttfamily,
  mathescape
}

\title{Análise e Projeto de Algoritmos -- Exercícios 02 (Grafos)}
\author{Prof. Thiago Cavalcante}
\date{}

\begin{document}

\pagenumbering{gobble}
\maketitle

\sloppy
\raggedright

\setlength{\leftmargini}{0pt}

\textbf{Para todas as questões, assuma que as listas de adjacência estão ordenadas por ordem crescente alfabética/numérica}.

\begin{enumerate}
  % SKIENA, S. S. The Algorithm Design Manual, 2nd Edition -- 5.31
  \item Quais são as estruturas de dados utilizadas na BFS e na DFS?

  %%%%%%%%%%%%%%%%%%%%%%%%%%%%%%%%%%%%%%%%%%%%%%%%
  % BFS                                          %
  %%%%%%%%%%%%%%%%%%%%%%%%%%%%%%%%%%%%%%%%%%%%%%%%
  % CORMEN, T. H.; LEISERSON, C. E.; RIVEST, R. L.; STEIN, C. Introduction to Algorithms, 3rd Edition -- 22.2-1
  \item Obtenha os valores de distância e parentesco para a execução da BFS no grafo dirigido a seguir, usando o vértice 3 como inicial.

  \begin{center}
    \includegraphics[width=\textwidth]{graph2.png}
  \end{center}

  % CORMEN, T. H.; LEISERSON, C. E.; RIVEST, R. L.; STEIN, C. Introduction to Algorithms, 3rd Edition -- 22.2-2
  \item Obtenha os valores de distância e parentesco para a execução da BFS no grafo não-dirigido a seguir, usando o vértice u como inicial.

  \begin{center}
    \includegraphics[width=\textwidth]{graph3.png}
  \end{center}

  % CORMEN, T. H.; LEISERSON, C. E.; RIVEST, R. L.; STEIN, C. Introduction to Algorithms, 3rd Edition -- 22.2-5
  \item Usando a figura da questão anterior, mostre como diferenças na ordem interna de cada lista de adjacência podem alterar a árvore de parentesco gerada pela BFS.

  %%%%%%%%%%%%%%%%%%%%%%%%%%%%%%%%%%%%%%%%%%%%%%%%
  % DFS                                          %
  %%%%%%%%%%%%%%%%%%%%%%%%%%%%%%%%%%%%%%%%%%%%%%%%
  % CORMEN, T. H.; LEISERSON, C. E.; RIVEST, R. L.; STEIN, C. Introduction to Algorithms, 3rd Edition -- Figure 22.5
  \item Obtenha os valores de tempo e parentesco e as classificações dos vértices para a execução da DFS no grafos dirigidos a seguir. Use o vértice s como inicial para o primeiro grafo e assuma que os vértices são processados em ordem alfabética no segundo grafo.

  \begin{center}
    \includegraphics[width=\textwidth]{graph4.png}
  \end{center}

  \begin{center}
    \includegraphics[width=\textwidth]{graph5.png}
  \end{center}

  % CORMEN, T. H.; LEISERSON, C. E.; RIVEST, R. L.; STEIN, C. Introduction to Algorithms, 3rd Edition -- 22.3-3
  \item Mostre a estrutura de parênteses gerada pela DFS para o grafo da questão 1.

  %%%%%%%%%%%%%%%%%%%%%%%%%%%%%%%%%%%%%%%%%%%%%%%%
  % ORDENAÇÃO TOPOLÓGICA                         %
  %%%%%%%%%%%%%%%%%%%%%%%%%%%%%%%%%%%%%%%%%%%%%%%%
  \item Faça a ordenação topológica dos grafos a seguir
  \bigskip

  % SKIENA, S. S. The Algorithm Design Manual, 2nd Edition -- 5.2
  \begin{center}
    \includegraphics[width=\textwidth]{graph1.png}
  \end{center}

  % CORMEN, T. H.; LEISERSON, C. E.; RIVEST, R. L.; STEIN, C. Introduction to Algorithms, 3rd Edition -- 22.4-1
  \begin{center}
    \includegraphics[width=\textwidth]{graph6.png}
  \end{center}

  %%%%%%%%%%%%%%%%%%%%%%%%%%%%%%%%%%%%%%%%%%%%%%%%
  % COMPONENTES FORTEMENTE CONECTADOS            %
  %%%%%%%%%%%%%%%%%%%%%%%%%%%%%%%%%%%%%%%%%%%%%%%%
  % CORMEN, T. H.; LEISERSON, C. E.; RIVEST, R. L.; STEIN, C. Introduction to Algorithms, 3rd Edition -- 22.5-2
  \item Identifique os componentes fortemente conectados no segundo grafo da questão 5 e mostre o grafo acíclico resultante.
\end{enumerate}
\end{document}
