\documentclass[10pt]{beamer}

\input{general/packages.tex}
\usetheme{metropolis}

\usepgfplotslibrary{dateplot}

\lstset{
  basicstyle=\ttfamily,
  mathescape
}
\input{general/info.tex}

\subtitle{Aula 4}
\date{22 de novembro de 2019}

\begin{document}

\maketitle

\begin{frame}{Avaliação AB1 - Opções (escolha 2)}
  \large
  \begin{itemize}
    \item Lista de exercícios

    \begin{itemize}
      \normalsize
      \item em equipes (nº equipes = nº questões)
      \item apresentação em sala (sorteio aluno e questão)
      \item data: 13/12
    \end{itemize}

    \item Desafios de programação do livro

    \begin{itemize}
      \normalsize
      \item em equipes
      \item apresentação em sala (sorteio aluno)
      \item submissão pela internet
      \item data: 13/12
    \end{itemize}

    \item Prova

    \begin{itemize}
      \normalsize
      \item individual
      \item presencial
      \item data: 06/12
    \end{itemize}
  \end{itemize}
\end{frame}

\begin{frame}{Estruturas de Dados}
  \huge
  \textbf{Fila de prioridade}
  \vfill
  \Large
  Funciona como uma fila normal onde cada elemento possui uma \textbf{prioridade} associada
\end{frame}

\begin{frame}{Estruturas de Dados}
  \huge
  \textbf{Fila de prioridade}
  \vfill
  \Large
  \begin{itemize}
    \item Inserção
    \item Encontrar Mínimo/Máximo
    \item Remover Mínimo/Máximo
  \end{itemize}
\end{frame}

\begin{frame}{Busca e Ordenação}
  \huge
  \textbf{Por que ordernar?}
  \vfill
  \large
  \begin{itemize}
    \item Ordenação é a \textbf{base para outros algoritmos}
    \item Várias \textbf{idéias de projeto de algoritmos} aparecem num contexto de ordenação: dividir-e-conquistar, estruturas de dados, algoritmos randomizados
    \item Computadores passam \textbf{mais tempo ordenando} do que fazendo qualquer outra coisa
    \item Ordenação é um dos problemas \textbf{mais estudados} em computação, com dezenas de algoritmos conhecidos para situações gerais ou específicas
  \end{itemize}
\end{frame}
\end{document}
