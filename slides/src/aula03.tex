\documentclass[10pt]{beamer}

\input{general/packages.tex}
\usetheme{metropolis}

\usepgfplotslibrary{dateplot}

\lstset{
  basicstyle=\ttfamily,
  mathescape
}
\input{general/info.tex}

\subtitle{Aula 3}
\date{08 de novembro de 2019}

\begin{document}

\maketitle

\begin{frame}{Estruturas de Dados}
  \huge
  \textbf{Tipos abstratos de dados}
  \vfill
  \begin{itemize}
    \item Contêineres (filas, pilhas)
    \item Dicionários
    \item Filas de prioridade
  \end{itemize}
\end{frame}

\begin{frame}{Estruturas de Dados}
  \huge
  Tipos abstratos de dados são implementados com \textbf{estruturas de dados}

\end{frame}

\begin{frame}{Estruturas de Dados}
  \huge
  \textbf{Estruturas contíguas $\times$ ligadas}
  \vfill
  \large
  \begin{columns}
    \begin{column}{0.49\textwidth}
      \begin{itemize}
        \item \textbf{Único} pedaço de memória
        \item Vetores (array 1D)
        \item Matrizes (array 2D)
        \item Heaps
        \item Tabelas de dispersão (\textit{hash tables})
      \end{itemize}

    \end{column}
    \begin{column}{0.49\textwidth}
      \begin{itemize}
        \item \textbf{Vários} pedaços de memória conectados por \textbf{ponteiros}
        \item Listas
        \item Árvores
        \item Listas de adjacência (grafos)
      \end{itemize}

    \end{column}
  \end{columns}
\end{frame}

\begin{frame}{Estruturas de Dados}
  \huge
  \textbf{Arrays}
  \vfill
  \LARGE
  Estruturas de \textbf{dados com tamanho fixo} onde cada elemento pode ser acessado de forma \textbf{eficicente} pelo seu \textbf{índice} (ou \textbf{endereço})

\end{frame}

\begin{frame}{Estruturas de Dados}
  \huge
  \textbf{Vantagens de um array}
  \vfill
  \Large
  \begin{itemize}
    \item Acesso aos elementos é $O(1)$, dado o índice
    \item Eficiência de espaço: composto puramente de dados com o mesmo tamanho
    \item Continuidade física dos dados na memória
  \end{itemize}

\end{frame}

\begin{frame}{Estruturas de Dados}
  \huge
  \textbf{Principal desvantagem de um array: tamanho fixo}
  \vfill
  \Large
  Possíveis soluções
  \begin{itemize}
    \item Alocar um array grande o suficiente (possível desperdício de memória)
    \item Arrays \textbf{dinâmicos} (ex.)
  \end{itemize}

\end{frame}

\begin{frame}{Estruturas de Dados}
  \huge
  \textbf{Ponteiros e estruturas ligadas}
  \vfill
  \Large
  \begin{itemize}
    \item Representam \textbf{endereços} na memória
    \item Oferecem maior \textbf{flexibilidade}
    \item Em C: operadores \textbf{\texttt{*}} e \textbf{\&}, valor \textbf{\texttt{NULL}}
  \end{itemize}

\end{frame}

\begin{frame}[fragile]{Estruturas de Dados}
  \huge
  \textbf{Implementação de uma lista}
  \vfill
  \large
  \begin{lstlisting}[language=C]
typedef struct lista {
  tipo dado; // -> dados
  Lista *prox; // -> ponteiro para o
} Lista;       //    proximo elemento
  \end{lstlisting}
\end{frame}

\begin{frame}{Estruturas de Dados}
  \huge
  \textbf{Características da lista ligada}
  \vfill
  \large
  \begin{itemize}
    \item Cada elemento armazena um ou mais dados
    \item Cada elemento possui também um \textbf{ponteiro para o próximo elemento} (ocupa mais espaço na memória)
    \item É necessário sempre o \textbf{ponteiro para o primeiro elemento} da lista
    \item Estrutura ligada mais simples
    \item Pode ser \textbf{duplamente} encadeada (ponteiros para o anterior e o próximo, ocupa mais espaço)
  \end{itemize}

\end{frame}

\begin{frame}[fragile]{Estruturas de Dados}
  \huge
  \textbf{Busca em uma lista}
  \vfill
  \large
  \begin{lstlisting}[language=C]
Lista *busca_lista(Lista *l, tipo x) {
  if (l == NULL) // lista vazia
    return(NULL);

  if (l->dado == x) // dado encontrado
    return(l);
  else // dado nao encontrado
    return(busca_lista(l->prox, x));
}
  \end{lstlisting}
\end{frame}

\begin{frame}[fragile]{Estruturas de Dados}
  \huge
  \textbf{Inserção em uma lista}
  \vfill
  \large
  \begin{lstlisting}[language=C]
void inserir_lista(Lista **l, tipo x) {
  Lista *p; // ponteiro auxiliar

  p = malloc(sizeof(Lista));
  p->dado = x;
  p->prox = *l;
  *l = p;
}
  \end{lstlisting}
\end{frame}

\begin{frame}[fragile]{Estruturas de Dados}
  \huge
  \textbf{Remoção em uma lista}
  \vfill
  \large
  \begin{lstlisting}[language=C]
Lista *anterior(Lista *l, tipo x) {
  if ((l == NULL) || (l->prox == NULL)) {
    printf("Nao encontrado");
    return(NULL);
  }
  if ((l->prox)->dado == x)
    return(l);
  else
    return(anterior(l->prox, x));
}
  \end{lstlisting}
\end{frame}

\begin{frame}[fragile]{Estruturas de Dados}
  \huge
  \textbf{Remoção em uma lista}
  \vfill
  \large
  \begin{lstlisting}[language=C]
void remover_lista(Lista **l, tipo x) {
  Lista *p, *pred;
  p = busca_lista(*l,x);
  if (p != NULL) {
    pred = anterior(*l,x);
    if (pred == NULL)
      *l = p->prox;
    else
      pred->prox = p->prox;
    free(p);
  }
}
  \end{lstlisting}
\end{frame}

\begin{frame}{Estruturas de Dados}
  \huge
  \textbf{Vantagens das listas}
  \vfill
  \large
  \begin{itemize}
    \item Uma lista nunca vai "estourar" a memória, a não ser que a memória do PC acabe
    \item Inserção e remoção são mais simples
    \item Gerenciamento de ponteiros é mais simples em larga escala
  \end{itemize}
\end{frame}

\begin{frame}{Estruturas de Dados}
  \huge
  \textbf{Desvantagens das listas}
  \vfill
  \large
  \begin{itemize}
    \item Precisam de mais espaço para os ponteiros
    \item Acesso aleatório não é eficiente
    \item Memória espalhada diminui a performance
  \end{itemize}
\end{frame}


\begin{frame}{Estruturas de Dados}
  \huge
  \textbf{Listas e arrays são estruturas \alert{recursivas}}
\end{frame}

\begin{frame}{Estruturas de Dados}
  \huge
  \textbf{Pilhas (stacks)}
  \vfill
  \Large
  \begin{itemize}
    \item \textit{last-in, first-out} (LIFO)
    \item Simples de implementar
    \item Eficientes
    \item Ordem não importa
    \item Operações: \texttt{push(p, x)} e \texttt{pop(p)}
    \item Aparecem em algoritmos recursivos
  \end{itemize}
\end{frame}

\begin{frame}{Estruturas de Dados}
  \huge
  \textbf{Filas (queues)}
  \vfill
  \Large
  \begin{itemize}
    \item \textit{first-in, first-out} (FIFO)
    \item Complicadas de implementar
    \item Minimizam o tempo máximo de espera
    \item Ordem importa
    \item Operações: \texttt{enqueue(f, x)} e \texttt{dequeue(p)}
  \end{itemize}
\end{frame}

\begin{frame}{Estruturas de Dados}
  \huge
  \textbf{Pilhas e filas podem ser implementadas tanto com arrays quanto com listas}
\end{frame}

\begin{frame}{Estruturas de Dados}
  \huge
  \textbf{Dicionários}
  \vfill
  \Large
  Armazenam um conjunto de \textbf{dados}, indexados por chaves.
  \vfill
  \large
  \texttt{\{"chave1": dado1, "chave2": dado2, $\cdots$ \}}

\end{frame}

\begin{frame}{Estruturas de Dados}
  \huge
  \textbf{Operações com dicionários}
  \vfill
  \Large
  \begin{itemize}
    \item \texttt{Busca(D, c)}
    \item \texttt{Inserção(D, x)}
    \item \texttt{Remoção(D, x)}
    \item \texttt{Máx(D) / Mín(D)}
    \item \texttt{Anterior(D, k) / Próximo(D, k)}
  \end{itemize}
  \vfill
  \large
  Exemplo: lista de elementos únicos
\end{frame}

\begin{frame}{Estruturas de Dados}
  \huge
  \textbf{Exercício: implementação de dicionário com array}
  \vfill
  Array ordenado $\times$ não ordenado
\end{frame}

\begin{frame}{Estruturas de Dados}
  \begin{center}
    \large
    \begin{tabular}{@{}lll@{}}
      \toprule
      \textbf{Operação} & \textbf{Array não ordenado} & \textbf{Array ordenado} \\
      \midrule
      Busca & \onslide<2->{$O(n)$} & \onslide<9->{$O(\log n)$} \\
      Inserção & \onslide<3->{$O(1)$} & \onslide<10->{$O(n)$} \\
      Remoção & \onslide<4->{$O(1)*$} & \onslide<11->{$O(n)$} \\
      Próximo & \onslide<5->{$O(n)$} & \onslide<12->{$O(1)$} \\
      Anterior & \onslide<6->{$O(n)$} & \onslide<13->{$O(1)$} \\
      Mín & \onslide<7->{$O(n)$} & \onslide<14->{$O(1)$} \\
      Máx & \onslide<8->{$O(n)$} & \onslide<15->{$O(1)$} \\
      \bottomrule
    \end{tabular}
  \end{center}
  \vfill
  \large
  Assumindo que o número \textbf{$n$} de elementos no array é dado
\end{frame}

\begin{frame}{Estruturas de Dados}
  \huge
  \textbf{Exercício: implementação de dicionário com lista}
  \vfill
  \Large
  Lista simplesmente ligada ordenada $\times$ lista simplesmente ligada não ordenada $\times$ lista duplamente ligada ordenada $\times$ lista duplamente ligada não ordenada
\end{frame}

\begin{frame}{Estruturas de Dados}
  \begin{center}
    \begin{tabular}{@{}lllll@{}}
      \toprule
      \textbf{Operação} & \textbf{Simp. não ord.} & \textbf{Dup. não ord.} & \textbf{Simp. ord.} & \textbf{Dup. ord.} \\
      \midrule
      Busca & \onslide<2->{$O(n)$} & \onslide<9->{$O(n)$} & \onslide<16->{$O(n)$} & \onslide<23->{$O(n)$} \\
      Inserção & \onslide<3->{$O(1)$} & \onslide<10->{$O(1)$} & \onslide<17->{$O(n)$} & \onslide<24->{$O(n)$} \\
      Remoção & \onslide<4->{$O(n)*$} & \onslide<11->{$O(1)$} & \onslide<18->{$O(n)*$} & \onslide<25->{$O(1)$} \\
      Próximo & \onslide<5->{$O(n)$} & \onslide<12->{$O(n)$} & \onslide<19->{$O(1)$} & \onslide<26->{$O(1)$} \\
      Anterior & \onslide<6->{$O(n)$} & \onslide<13->{$O(n)$} & \onslide<20->{$O(n)*$} & \onslide<27->{$O(1)$} \\
      Mín & \onslide<7->{$O(n)$} & \onslide<14->{$O(n)$} & \onslide<21->{$O(1)$} & \onslide<28->{$O(1)$} \\
      Máx & \onslide<8->{$O(n)$} & \onslide<15->{$O(n)$} & \onslide<22->{$O(1)*$} & \onslide<29->{$O(1)$} \\
      \bottomrule
    \end{tabular}
  \end{center}
\end{frame}

\begin{frame}{Estruturas de Dados}
  \huge
  \textbf{Árvores binárias de busca}
  \vfill
  \large
  \begin{itemize}
    \item Até agora: busca rápida ou atualização flexível, não ambas
    \item Busca binária requer acesso aos nós medianos \textbf{acima} e \textbf{abaixo} do que se procura
    \item Árvore binária de busca é uma "lista ligada" com dois ponteiros por elemento
  \end{itemize}

\end{frame}

\begin{frame}{Estruturas de Dados}
  \huge
  \textbf{Definição recursiva de árvore binária}
  \vfill
  \Large
  \begin{enumerate}
    \item Vazia, ou
    \item Elemento chamado de \textbf{raiz} com duas outras árvores binárias, chamadas \textbf{subárvores} da \textbf{esquerda} e da \textbf{direita}
  \end{enumerate}
  \vfill
  \large
  A ordem dos elementos importa
\end{frame}

\begin{frame}{Estruturas de Dados}
  \huge
  \textbf{Em uma árvore binária \alert{de busca}, todos os nós na \alert{subárvore da esquerda são menores} que a raiz e todos os nós na \alert{subárvore da direita são maiores} que a raiz}
  \vfill
  \large
  Com uma árvore binária de \textbf{$n$} elementos e um conjunto de \textbf{$n$} valores, só existe \textbf{uma maneira} de criar uma árvore binária de busca
\end{frame}

\begin{frame}[fragile]{Estruturas de Dados}
  \huge
  \textbf{Implementação de uma árvore}
  \vfill
  \large
  \begin{lstlisting}[language=C]
typedef struct arvore {
  tipo dado;
  struct Arvore *pai; // opcional
  struct Arvore *esquerda;
  struct Arvore *direita;
} Arvore;
  \end{lstlisting}
\end{frame}

\begin{frame}{Estruturas de Dados}
  \huge
  \textbf{Operações básicas}
  \vfill
  \LARGE
  \begin{itemize}
    \item Busca
    \item Travessia
    \item Inserção
    \item Remoção
  \end{itemize}
\end{frame}

\begin{frame}[fragile]{Estruturas de Dados}
  \huge
  \textbf{Busca em uma árvore}
  \vfill
  \large
  \begin{lstlisting}[language=C]
Arvore *busca_arvore(Arvore *l, tipo x) {
  if (l == NULL) return(NULL);
  if (l->dado == x) return(l);
  if (x < l->dado)
    return(busca_arvore(l->esquerda, x));
  else
    return(busca_arvore(l->direita, x));
}
  \end{lstlisting}
  \vfill
  \large
  Roda em tempo \textbf{$O(h)$}, onde \textbf{$h$} é a altura da árvore
\end{frame}

\begin{frame}[fragile]{Estruturas de Dados}
  \huge
  \textbf{Mínimo/máximo em uma árvore}
  \vfill
  \large
  \begin{lstlisting}[language=C]
Arvore *minimo(Arvore *t) {
  Arvore *min; // ponteiro auxiliar
  if (t == NULL) return(NULL);
  min = t;
  while (min->esquerda != NULL)
    min = min->esquerda;
  return(min);
}
  \end{lstlisting}
\end{frame}

\begin{frame}[fragile]{Estruturas de Dados}
  \huge
  \textbf{Atravessando uma árvore}
  \vfill
  \large
  \begin{lstlisting}[language=C]
void atravessar_arvore(Arvore *l) {
  if (l != NULL) {
    atravessar_arvore(l->esquerda);
    processar_elemento(l->dado);
    atravessar_arvore(l->direita);
  }
}
  \end{lstlisting}
  \vfill
  \large
  A travessia ocorre em \textbf{ordem ascendente} dos dados, em um tempo \textbf{$O(n)$} (cada um dos \textbf{$n$} elementos é processado uma vez)
\end{frame}

\begin{frame}[fragile]{Estruturas de Dados}
  \huge
  \textbf{Inserção em uma árvore}
  \vfill
  \LARGE
  Realizada no ponto onde a busca pelo elemento que se deseja inserir falha (encontra um elemento \texttt{\textbf{NULL}})
  \vfill
  \large
  Roda em tempo \textbf{$O(h)$}, onde \textbf{$h$} é a altura da árvore
\end{frame}

\begin{frame}[fragile]{Estruturas de Dados}
  \vfill
  \begin{lstlisting}[language=C]
insere_arvore(Arvore **l, tipo x, Arvore *pai) {
  Arvore *p; // ponteiro auxiliar
  if (*l == NULL) {
    p = malloc(sizeof(Arvore)); // alocacao
    p->dado = x;
    p->esquerda = p->direita = NULL;
    p->pai = pai;
    *l = p; // conectando a arvore
    return;
  }
  if (x < (*l)->dado)
    insere_arvore(&((*l)->esquerda), x, *l);
  else
    insere_arvore(&((*l)->direita), x, *l);
}
  \end{lstlisting}
\end{frame}

\begin{frame}{Estruturas de Dados}
  \huge
  \textbf{Remoção em uma árvore}
  \vfill
  \LARGE
  Três possibilidades
  \begin{enumerate}
    \item Elemento não tem filhos
    \item Elemento tem um filho
    \item Elemento tem dois filhos (complexo)
  \end{enumerate}
  \vfill
  \large
  Roda em tempo \textbf{$O(h)$}, onde \textbf{$h$} é a altura da árvore
\end{frame}

\begin{frame}{Estruturas de Dados}
  \huge
  \textbf{Pior caso}
  \vfill
  Altura \textbf{$h$} é igual ao número \textbf{$n$} de elementos, árvore vira uma lista encadeada e todos os algoritmos rodam no tempo \textbf{$O(n)$}
\end{frame}

\begin{frame}{Estruturas de Dados}
  \huge
  \textbf{Melhor caso}
  \vfill
  A árvore é \textbf{perfeitamente balanceada} (dois filhos por elemento), a altura é igual a \textbf{$\lceil \log_2 n \rceil$} e todos os algoritmos rodam no tempo \textbf{$O(\log_2 n)$}
\end{frame}

\begin{frame}{Estruturas de Dados}
  \huge
  \textbf{Na \alert{média}, existe uma \alert{alta probabilidade} de a árvore ter altura \textbf{$O(\log_2 n)$}}
\end{frame}

\begin{frame}{Estruturas de Dados}
  \huge
  \textbf{Exercício: lendo $n$ números e imprimindo na ordem com um dicionário balanceado}
  \vfill
  \large
  \begin{enumerate}
    \item Usando \texttt{inserção} e \texttt{travessia}
    \item Usando \texttt{inserção}, \texttt{mínimo} e \texttt{próximo}
    \item Usando \texttt{inserção}, \texttt{mínimo} e \texttt{remoção}
  \end{enumerate}
\end{frame}
\end{document}
