\documentclass[10pt]{beamer}

\input{general/packages.tex}
\usetheme{metropolis}

\usepgfplotslibrary{dateplot}

\lstset{
  basicstyle=\ttfamily,
  mathescape
}
\input{general/info.tex}

\subtitle{Aula 6}
\date{6 de dezembro de 2019}

\begin{document}

\maketitle

\begin{frame}{Grafos}
  \huge
  \textbf{Representação abstrata para descrever:}
  \vfill
  \Large
  \begin{itemize}
    \item Sistemas de transporte
    \item Interações humanas
    \item Redes de telecomunicações
  \end{itemize}
  \vfill
  \large
  Uma mesma definição serve para modelar várias estruturas
\end{frame}

\begin{frame}{Grafos}
  \Huge
  \centering
  $G = (V, E)$
  \vfill
  \LARGE
  \[
    \left\{
      \begin{array}{ll}
        V = & \text{conjunto de vértices}\\
        E = & \text{conjunto de pares de vértices} \\
            & \text{ou arestas}
      \end{array}
    \right.
  \]
\end{frame}

\setbeamertemplate{frame footer}{Figura 5.1, p. 145 (PDF: p. 157)}
\begin{frame}{Grafos}
  \Large
  Grafos podem modelar essencialmente \textbf{qualquer} relação
  \vfill
  \large
  \begin{itemize}
    \item Redes de rodovias, onde os vértices são cidades e as arestas são as estradas
    \item Circuitos elétricos, onde os vértices são as junções e as arestas são os componentes
    \item Código-fonte, onde os vértices são as linhas de código e as arestas representam o fluxo entre elas
    \item Interações humanas, onde os vértices são as pessoas e as arestas representam conexões
  \end{itemize}
\end{frame}
\setbeamertemplate{frame footer}{}

\begin{frame}{Grafos}
  \huge
  A chave para a solução de muitos problemas de algoritmos está em interpretá-los \textbf{em termos de grafos}
  \vfill
  \large
  ``Linguagem'' para descrever propriedades das relações (problemas complexos $\rightarrow$ descrição e soluções simples em termos de grafos)
\end{frame}

\begin{frame}{Grafos}
  \Large
  \begin{itemize}
    \item Criar novos algoritmos de grafos é uma \textbf{tarefa complexa}
    \item Melhor forma: modelar o problema de forma que um \textbf{algoritmo já existente} possa ser utilizado
    \item É importante se familiarizar com problemas de algoritmos com grafos (cada problema tem um algortimo mais adequado)
  \end{itemize}
\end{frame}

\setbeamertemplate{frame footer}{Figura 5.2, p. 147 (PDF: p. 158)}
\begin{frame}{Tipos de Grafos}
  \vfill
  \centering
  \huge \textbf{Dirigido $\times$ Não-dirigido}
  \vfill
  \raggedright
  \large
  \begin{itemize}
    \item Não-dirigido: para um grafo $G = (V, E)$, a aresta $(x, y) \in E$ implica $(y, x) \in E$
    \item Exemplo: rodovias $\times$ ruas
    \item Maioria dos grafos de interesse teórico são não-dirigidos
  \end{itemize}
\end{frame}

\begin{frame}{Tipos de Grafos}
  \vfill
  \centering
  \huge \textbf{Ponderado $\times$ Não-ponderado}
  \vfill
  \raggedright
  \large
  \begin{itemize}
    \item Em um grafo ponderado, cada aresta possui um valor numérico associado chamado de peso (pode introduzir um cálculo de custo)
    \item Exemplo: rodovias
    \item Exemplo: caminho mais curto
  \end{itemize}
\end{frame}

\begin{frame}{Tipos de Grafos}
  \vfill
  \centering
  \huge \textbf{Simples $\times$ Multigrafo}
  \vfill
  \raggedright
  \large
  \begin{itemize}
    \item Um grafo simples não possui laços ou múltiplas arestas conectando os mesmos vértices
    \item Essas conexões requerem um cuidado especial na hora da implementação de um algoritmo
    \item Em geral, ``grafo'' = ``grafo simples''
  \end{itemize}
\end{frame}

\begin{frame}{Tipos de Grafos}
  \vfill
  \centering
  \huge \textbf{Esparso $\times$ Denso}
  \vfill
  \raggedright
  \large
  \begin{itemize}
    \item Definição vaga
    \item Em geral, os grafos esparsos possuem uma quantidade de arestas proporcional a $V$ e os grafos densos possuem uma quantidade proporcional a $V^2$
    \item Exemplos: ruas e circuitos
  \end{itemize}
\end{frame}

\begin{frame}{Tipos de Grafos}
  \vfill
  \centering
  \huge \textbf{Cíclico $\times$ Acíclico}
  \vfill
  \raggedright
  \large
  \begin{itemize}
    \item Um grafo acíclico não contém nenhum ciclo
    \item Exemplo: árvores e DAGs (agendamento)
  \end{itemize}
\end{frame}

\begin{frame}{Tipos de Grafos}
  \vfill
  \centering
  \huge \textbf{Embutido $\times$ Topológico}
  \vfill
  \raggedright
  \large
  \begin{itemize}
    \item Qualquer grafo para o qual são atribuídas posições geométricas aos vértices e arestas é um grafo embutido
    \item Um grafo topológico é definido pela ``regra'' subjacente
    \item Exemplos: caixeiro-viajante, grade de pontos
  \end{itemize}
\end{frame}

\begin{frame}{Tipos de Grafos}
  \vfill
  \centering
  \huge \textbf{Implícito $\times$ Explícito}
  \vfill
  \raggedright
  \large
  \begin{itemize}
    \item Um grafo implícito é construído a medida que vai sendo utilizado
  \end{itemize}
\end{frame}

\begin{frame}{Tipos de Grafos}
  \vfill
  \centering
  \huge \textbf{Rotulado $\times$ Não-rotulado}
  \vfill
  \raggedright
  \large
  \begin{itemize}
    \item Cada vértice em um grafo rotulado possui um nome ou identificador único, para distinguí-lo de todos os outros vértices
    \item Exemplo: cidades
  \end{itemize}
\end{frame}
\setbeamertemplate{frame footer}{}

\setbeamertemplate{frame footer}{Figura 5.3, p. 149 (PDF: p. 161)}
\begin{frame}{Grafo da amizade}
  \LARGE
  \begin{itemize}
    \item Exemplo de modelagem
    \item Vértices: pessoas
    \item Arestas: relação de amizade
    \item Esses grafos são chamados de redes sociais
  \end{itemize}
\end{frame}

\begin{frame}{Grafo da amizade}
  \huge
  \textbf{Direção}: se eu sou seu amigo, isso significa que você é meu amigo? (outro exemplo: grafo de ``ouvir falar'')
\end{frame}

\begin{frame}{Grafo da amizade}
  \huge
  \textbf{Peso}: o quão próximo você é do seu amigo?
\end{frame}

\begin{frame}{Grafo da amizade}
  \huge
  \textbf{Simplicidade}: eu sou meu próprio amigo? (multigrafo: diferentes relações de amizade)
\end{frame}

\begin{frame}{Grafo da amizade}
  \huge
  \textbf{Densidade}: quem tem mais amigos?
\end{frame}

\begin{frame}{Grafo da amizade}
  \huge
  \textbf{Geometria}: meus amigos moram perto de mim? (redes sociais requerem um grafo embutido)
\end{frame}

\begin{frame}{Grafo da amizade}
  \huge
  \textbf{Implícito/Explícito}: você também conhece aquela pessoa? (sites de redes sociais $\times$ vida real)
\end{frame}

\begin{frame}{Grafo da amizade}
  \huge
  \textbf{Rótulos}: você é um indivíduo ou parte de uma multidão? (em geral, estudo de redes sociais não se importa com nomes)
  \vfill
  \large
\end{frame}
\setbeamertemplate{frame footer}{}

\begin{frame}{Estruturas de dados para grafos}
  \huge
  \begin{itemize}
    \item Matriz de adjacência
    \item Lista de adjacência
  \end{itemize}
\end{frame}

\begin{frame}{Estruturas de dados para grafos}
  \huge
  \textbf{Matriz de adjacência}
  \vfill
  \large
  Podemos representar um grafo $G$ com uma matriz $M$ de tamanho $n \times n$ (onde $n$ é o número de vértices), de forma que $M(i, j) = 1$ se a aresta $(i, j)$ é parte de $G$ e $M(i, j) = 0$, caso contrário

  Essa estrutura permite uma rápida resposta para a pergunta ``a aresta $(i, j)$ está presente em $G$?'', e também uma rápida atualização do grafo (inserção e remoção de arestas)

  Pode desperdiçar espaço em um gráfico com muitos vértices e poucas arestas
\end{frame}

\begin{frame}{Estruturas de dados para grafos}
  \huge
  \textbf{Lista de adjacência}
  \vfill
  \large
  Grafos esparsos podem ser representado de forma mais eficiente com listas encadeadas, as quais armazenam os vértices vizinhos de um determinado vértice
\end{frame}

\begin{frame}{Estruturas de dados para grafos}
  \huge
  \textbf{Implementação da lista de adjacência}
\end{frame}

\begin{frame}{Atravessando um grafo}
  \huge
  \textbf{Problema: visitar cada aresta e cada vértice em um grafo de forma sistemática}
\end{frame}

\begin{frame}{Atravessando um grafo}
  \huge
  \textbf{Comparação: grafo e labirinto}
  \vfill
  \large
  Um labirinto pode ser naturalmente representado por um grafo, onde os vértices denotam uma junção do labirinto e as arestas denotam os caminhos. Sendo assim, um algoritmo para atravessar um grafo deve ser poderoso o suficiente para nos tirar de um labirinto.
\end{frame}

\begin{frame}{Atravessando um grafo}
  \huge
  \textbf{Comparação: grafo e labirinto}
  \vfill
  \Large
  \begin{itemize}
    \item \textbf{Eficiência}: evitar visitar os mesmos lugares repetidamente
    \item \textbf{Correção}: travessia sistemática que garanta a saída do labirinto, cada vértice e aresta devem ser visitados
  \end{itemize}
\end{frame}

\begin{frame}{Atravessando um grafo}
  \huge
  \textbf{Idéia principal: categorizar cada vértice como}
  \vfill
  \Large
  \begin{itemize}
    \item Não descoberto: estado inicial
    \item Descoberto: foi encontrado, mas suas arestas não foram checadas
    \item Processado: foi encontrado e todas as suas arestas foram visitadas
  \end{itemize}
\end{frame}
\end{document}
